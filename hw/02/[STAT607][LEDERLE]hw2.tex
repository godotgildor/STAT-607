\documentclass{article}\usepackage{graphicx, color}
%% maxwidth is the original width if it is less than linewidth
%% otherwise use linewidth (to make sure the graphics do not exceed the margin)
\makeatletter
\def\maxwidth{ %
  \ifdim\Gin@nat@width>\linewidth
    \linewidth
  \else
    \Gin@nat@width
  \fi
}
\makeatother

\IfFileExists{upquote.sty}{\usepackage{upquote}}{}
\definecolor{fgcolor}{rgb}{0.2, 0.2, 0.2}
\newcommand{\hlnumber}[1]{\textcolor[rgb]{0,0,0}{#1}}%
\newcommand{\hlfunctioncall}[1]{\textcolor[rgb]{0.501960784313725,0,0.329411764705882}{\textbf{#1}}}%
\newcommand{\hlstring}[1]{\textcolor[rgb]{0.6,0.6,1}{#1}}%
\newcommand{\hlkeyword}[1]{\textcolor[rgb]{0,0,0}{\textbf{#1}}}%
\newcommand{\hlargument}[1]{\textcolor[rgb]{0.690196078431373,0.250980392156863,0.0196078431372549}{#1}}%
\newcommand{\hlcomment}[1]{\textcolor[rgb]{0.180392156862745,0.6,0.341176470588235}{#1}}%
\newcommand{\hlroxygencomment}[1]{\textcolor[rgb]{0.43921568627451,0.47843137254902,0.701960784313725}{#1}}%
\newcommand{\hlformalargs}[1]{\textcolor[rgb]{0.690196078431373,0.250980392156863,0.0196078431372549}{#1}}%
\newcommand{\hleqformalargs}[1]{\textcolor[rgb]{0.690196078431373,0.250980392156863,0.0196078431372549}{#1}}%
\newcommand{\hlassignement}[1]{\textcolor[rgb]{0,0,0}{\textbf{#1}}}%
\newcommand{\hlpackage}[1]{\textcolor[rgb]{0.588235294117647,0.709803921568627,0.145098039215686}{#1}}%
\newcommand{\hlslot}[1]{\textit{#1}}%
\newcommand{\hlsymbol}[1]{\textcolor[rgb]{0,0,0}{#1}}%
\newcommand{\hlprompt}[1]{\textcolor[rgb]{0.2,0.2,0.2}{#1}}%

\usepackage{framed}
\makeatletter
\newenvironment{kframe}{%
 \def\at@end@of@kframe{}%
 \ifinner\ifhmode%
  \def\at@end@of@kframe{\end{minipage}}%
  \begin{minipage}{\columnwidth}%
 \fi\fi%
 \def\FrameCommand##1{\hskip\@totalleftmargin \hskip-\fboxsep
 \colorbox{shadecolor}{##1}\hskip-\fboxsep
     % There is no \\@totalrightmargin, so:
     \hskip-\linewidth \hskip-\@totalleftmargin \hskip\columnwidth}%
 \MakeFramed {\advance\hsize-\width
   \@totalleftmargin\z@ \linewidth\hsize
   \@setminipage}}%
 {\par\unskip\endMakeFramed%
 \at@end@of@kframe}
\makeatother

\definecolor{shadecolor}{rgb}{.97, .97, .97}
\definecolor{messagecolor}{rgb}{0, 0, 0}
\definecolor{warningcolor}{rgb}{1, 0, 1}
\definecolor{errorcolor}{rgb}{1, 0, 0}
\newenvironment{knitrout}{}{} % an empty environment to be redefined in TeX

\usepackage{alltt}
\usepackage[sc]{mathpazo}
\usepackage{geometry}
\geometry{verbose,tmargin=2.5cm,bmargin=2.5cm,lmargin=2.5cm,rmargin=2.5cm}
\setcounter{secnumdepth}{2}
\setcounter{tocdepth}{2}
\usepackage{url}
\usepackage[unicode=true,pdfusetitle, bookmarks=true,bookmarksnumbered=true,bookmarksopen=true,bookmarksopenlevel=2, breaklinks=false,pdfborder={0 0 1},backref=false,colorlinks=false] {hyperref}
\hypersetup{ pdfstartview={XYZ null null 1}}
\usepackage{breakurl}
\parindent = 0pt

\usepackage{amsmath}
\usepackage{framed, color}
\definecolor{shadecolor}{RGB}{211, 211, 211}

\title{Mike Lederle\\lederle@neo.tamu.edu\\STAT 607 --- HW 2}

\begin{document}
\maketitle
\section*{Exercise 2.12}
 The percentage of patients overdue for a vaccination is often of interest for a medical clinic. Some clinics examine every record to determine that percentage; in a large practice, though, taking a census can be time consuming. Cullen (1994) took a sample of the 580 children served by an Auckland family practice to estimate the proprtion of interest.
\begin{enumerate}
  \item[{\bf a}] What sample size in an SRS (without replacement) would be necessary to estimate the proprtion with 95\% confidence and margin of error 0.10?
\begin{shaded}
For $\alpha = .05$ and $e = .10$, we need 
$$
n_0 = \frac{z_{\alpha/2}^2 S^2}{e^2} = \frac{1.96^2 \cdot 1/2 \cdot (1 - 1/2)}{.10^2}
$$
\begin{knitrout}
\definecolor{shadecolor}{rgb}{0.969, 0.969, 0.969}\color{fgcolor}\begin{kframe}
\begin{alltt}
(n_0 <- (1.96^2 * 0.5^2)/0.1^2)
\end{alltt}
\begin{verbatim}
## [1] 96.04
\end{verbatim}
\end{kframe}
\end{knitrout}

The final calculation is
$$
n = \frac{n_0}{1 + \frac{n_0}{N}},
$$
which is 
\begin{knitrout}
\definecolor{shadecolor}{rgb}{0.969, 0.969, 0.969}\color{fgcolor}\begin{kframe}
\begin{alltt}
N <- 580
(n <- n_0/(1 + n_0/N))
\end{alltt}
\begin{verbatim}
## [1] 82.4
\end{verbatim}
\end{kframe}
\end{knitrout}

\end{shaded}
\vfil
\pagebreak
\item[{\bf b}] Cullen actually took a SRSWR of size 120, of which
27 were {\sl not} overdue for vaccination. Give a 95\% CI for the
proprtion of children not overdue for vaccination.
\begin{shaded}
the sample proprtion is $\hat{p} = 27/120$
\begin{knitrout}
\definecolor{shadecolor}{rgb}{0.969, 0.969, 0.969}\color{fgcolor}\begin{kframe}
\begin{alltt}
(p.hat <- 27/120)
\end{alltt}
\begin{verbatim}
## [1] 0.225
\end{verbatim}
\end{kframe}
\end{knitrout}

The standard error for $\hat{p}$ is
$$
\sqrt{\left(1 - \frac{n}{N}\right)\frac{\hat{p}(1 - \hat{p})}{n-1}}
$$
\begin{knitrout}
\definecolor{shadecolor}{rgb}{0.969, 0.969, 0.969}\color{fgcolor}\begin{kframe}
\begin{alltt}
(se <- \hlfunctioncall{sqrt}((1 - n/N) * (p.hat * (1 - p.hat)/(n - 1))))
\end{alltt}
\begin{verbatim}
## [1] 0.04287
\end{verbatim}
\end{kframe}
\end{knitrout}

The left and right end points of the CI:
\begin{knitrout}
\definecolor{shadecolor}{rgb}{0.969, 0.969, 0.969}\color{fgcolor}\begin{kframe}
\begin{alltt}
p.hat - 1.96 * se
\end{alltt}
\begin{verbatim}
## [1] 0.141
\end{verbatim}
\begin{alltt}
p.hat + 1.96 * se
\end{alltt}
\begin{verbatim}
## [1] 0.309
\end{verbatim}
\end{kframe}
\end{knitrout}


\end{shaded}
\end{enumerate}

\end{document}
